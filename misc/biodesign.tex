\documentclass[11pt]{report}
\usepackage{geometry}
\geometry{letterpaper, top=1in, bottom=1in, left=1in, right=1in}                                 
\usepackage[parfill]{parskip}   
\usepackage{graphicx}
\usepackage{amssymb}
\usepackage{epstopdf}
\usepackage{array}
\usepackage{tabularx}
\usepackage{hyperref}
\hypersetup{
    colorlinks=true,
    linkcolor=blue,
    filecolor=magenta,      
    urlcolor=cyan
    }


\setlength{\parindent}{0em}
\pagenumbering{gobble}

\DeclareGraphicsRule{.tif}{png}{.png}{`convert #1 `dirname #1`/`basename #1 .tif`.png}

\newcommand{\newday}[1]{\newpage{\LARGE \textbf{#1}}\\}
\newcommand{\imagefolder}{/Users/lauriaclarke/Documents/mfadt/ms1/lauriaclarke.github.io/instructionsetsforstrangers}


\title{Major Studio I \\ Final Project}
\date{\today}
\author{Lauria Clarke}
                                   
\begin{document}
\maketitle
%\tableofcontents
%----------------------------------------------
                                                                                                                                                                                                                                                                                                                                                                                                                                                                                                                                                                                                                                                                                                                              \tx
\textbf{Why you are interested in taking this class and participating in the Biodesign Challenge?}                                                                                                                                                                                                                                                                                                                                                                                                                                                   
I would like to take this course and participate in the biodesign challenge because I am trying to build a foundation of thought in the areas of biodesign and bioart. My prior background is in computer engineering and kinetic sculpture and I have always been fascinated by the field of biomimetics. 

I am perpetually drawn to this field because of the ambiguity and contrast it highlights -- blurring the line between what we intuitively know to be natural and what we intuitively know to be man-made or artificial. Recently, I have been trying to pin down the importance and utility of this particular uncertainty in a broader research contex. The root of my curiosity in addressing the boundary between what is natural and artificial is a desire to understand how creative practitioners can use it to generate reflection and consideration of humanity's place in the natural world. 

To conceptualize the impact of human activity on this planet is practically impossible, yet we find ourselves at a juncture where this definition is at the heart of nearly every social and political issue. If Art is a trascendent mechanism for collapsing the scales of Time and space we need it now more than ever to make sense of this recursively enormous question.  

While I don't have a specific project idea to contribute to this course right now, I am extremely curious to observe and be a part of the community of designers working at the very heart of these questions. Broadening my vocabulary and foundation of thought to address the above feels critical to the developement of my practice right now. 

\textbf{Describe your Science/Biology and Biodesign/Bioart experience or background.}

I am fascinated by naturalistic movement and much of my sculpture based work seeks to mimic naturalistic forms and motion. Outside of the biology I learned during primary and secondary school education, my biology background is entirely intuitive and learned from life spent watching very, very closely while playing outside.


\textbf{If you don’t already have a project and/or team, what are some of the areas of interest you have for investigation, exploration, research and/or intervention/engagement?}

As mentioned above, I am curious to explore the boundary of what we consider to be artificial and natural. Kinetic sculpture is the vocabulary I am comfortable using to address this in my own practice, but I'm eager to colaborate with others and push myself toward new language and modes of inquiry.  



-- how do yuo define the recursively enormous -- 

\end{document}


