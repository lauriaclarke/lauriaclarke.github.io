\documentclass[11pt]{report}
\usepackage{geometry}
\geometry{letterpaper, top=1in, bottom=1in, left=1in, right=1in}                                 
\usepackage[parfill]{parskip}   
\usepackage{graphicx}
\usepackage{amssymb}
\usepackage{epstopdf}
\usepackage{array}
\usepackage{tabularx}
\usepackage{hyperref}
\hypersetup{
    colorlinks=true,
    linkcolor=blue,
    filecolor=magenta,      
    urlcolor=cyan
    }


\setlength{\parindent}{0em}
\pagenumbering{gobble}

\DeclareGraphicsRule{.tif}{png}{.png}{`convert #1 `dirname #1`/`basename #1 .tif`.png}

\newcommand{\newday}[1]{\newpage{\LARGE \textbf{#1}}\\}
\newcommand{\imagefolder}{/Users/lauriaclarke/Documents/mfadt/ms1/lauriaclarke.github.io/instructionsetsforstrangers}


\title{INSTRUCTION SETS FOR STRANGERS}
\date{}
\author{Lauria Clarke}
                                   
\begin{document}
\maketitle
%\tableofcontents
%----------------------------------------------
\section*{PART I - Ideation}
\subsection*{Location Choice}

After some discussion, we decided that a crosswalk would be the most exciting place to observe interactions between strangers. Given the high degree of randomness in who you encounter while crossing the street it seemed like an interesting choice -- particularly if the goal is to foster interaction between strangers. 

To pick a crosswalk we walked along 17th street between 5th and 2nd avenues. 

On the northeast corner of Union Square Park we discovered a crosswalk that included a median area. Few crosswalks in Manhattan cross two-way traffic -- this crosswalk crosses Park Ave. which is two lanes of divided and opposing traffic. Overall, this has the effect of making the intersection much larger and the act of walking across the street is much longer. 

\begin{figure}[ht]
\centering
\includegraphics[scale=0.10, angle=270]{"images/I/crosswalk_nopeople.jpg"}
\caption{The selected crosswalk on 17th st.}
\label{fig:nopeople}
\end{figure}

% insert history about Union Sq Park here

\subsection*{Initial Observation}

After walking by the first time around 4 PM on a weekday, we decided to return on a Saturday at noon -- during the Farmer's Market. The area surrounding Union Square is heavily trafficked on weekdays by commuters going to or from the train. Our hope for the weekend, was that more people would be paying attention to their surroundings and less focused on getting where they needed to go quickly. After completing the AEIOU exercise at the intersection, this generally seemed to be true. Most people crossing the street were shopping at the market and spending time with friends. 

\begin{figure}[ht]
\centering
\includegraphics[scale=0.8]{"images/I/ms1_aeiou.png"}\\
\vspace{0.5cm}
\includegraphics[scale=0.8]{"images/I/ms1_behaviormap.png"}
\caption{AEIOU and Behavior mapping exercise}
\label{fig:behaviormap}
\end{figure}

Behavior mapping at the crosswalk revealed some more interesting data. People waiting at the crosswalk were generally found to be doing one of three things: chatting with a friend, looking at their phone, or looking at traffic. The majority of people waiting at the intersection spent their time looking at traffic. I think this may be due to the fact that this intersection is quite long -- it has a median -- and cars moving down Park Ave. tend to be going pretty fast. It seems like an intersection that requires attention as a pedestrian.  

Since most of the people crossing the street here were travelling to or from the farmer's market and many were with fiends or family, the majority of foot traffic was routed over the small delta in the bike lane (see shaded shape in map) with people clustered in groups. In comparison, few people stopped on the medians in Park Ave. Those that did were generally alone and walking out of sync with the lights and were paying attention to traffic, not their cellphones.


\subsection*{Ideation}

After collecting data, we began to brainstorm about which interactions would be most interesting between strangers crossing the street. The idea of playing a simple game with a strange on the other side of the crosswalk was very appealing, but most required more synchronization and direction than was possible in such a short amount of time (the amount of time spent waiting at the crosswalk).
 
\begin{figure}[ht]
\centering
\includegraphics[width=1.4\textwidth, angle=90]{"images/I/ms1_brainstorming.png"}
\caption{Results of brainstorming activity.}
\label{fig:brainstorm}
\end{figure}

After deciding that the ideal interaction would be really quick -- such as a greeting -- we began thinking about fun ways to wave at someone on the other side of the street. The idea of having a giant arm and hand on each side of the street was very appealing and seemed both simple and effective. The below sketch illustrates what a pair of hands / arms would look like on location.
 
\begin{figure}[ht]
\centering
\includegraphics[width=0.8\textwidth]{"images/I/wavemachine_overlay_2.jpg"}
\caption{Sketch of device on location.}
\label{fig:overlay}
\end{figure}

The best part of such a contraption is that it would not necessarily require the use of electronics. Using thin plywood, springs and some clever balancing, we could create the arm mechanism from very basic materials. Below is a sketch of one possible mechanism.
 
\begin{figure}[ht]
\centering
\includegraphics[width=0.8\textwidth]{"images/I/wavemachine_sketch.png"}
\caption{Possible mechanical design of device.}
\label{fig:sketch}
\end{figure}

\clearpage
\section*{PART II - Iteration}
\subsection*{prototyping}

The initial prototype was extremely quick. It successfully proved that the basic mechanism did, in fact work. \href{https://drive.google.com/file/d/1A-JF5d3mOUxOp_nL0GExQ2gzeuzpgY3x/view?usp=sharing}{Link to video of initial prototype.}

\begin{figure}[ht!]
\centering
\includegraphics[width=0.8\textwidth, angle=-90]{"images/II/prototype_1.JPG"}
\caption{Initial prototype.}
\end{figure}

The initial prototype required improvement in four areas; a more solid connection between the arm and the base, a way for the base to be connected to a street post securely, improved elastic, and an eye catching handle.

Using a bolt and some washers the arm and the base were connected in a way that allowed the arm to swing freely without twisting. Next, in order to mount the wave machine, we found some large zip ties and drilled holes in the base to thread the zip ties through. During this process we discovered that making the mechanism one-sided decreased the complexity of the device and made it easier to mount the base. By restricting the movement to be one-sided we also decreased the chance of the arm becoming stuck in either direction. Despite the reduction in range, it was still quite satisfying to use the mechanism. As we played with the movement of the arm, the easiest elastics on hand were hair-ties. By trying a couple different sizes we honed in on the best tension for the weight of the arm. Finally, we cut the end off of an aluminum mop and colored it bright red with a sharpie in the hope that this would grab someone's attention. \href{https://drive.google.com/file/d/10h-Otp0xxu8sbUBII9ftw7f-XZoaok2H/view?usp=sharing}{Link to video of second prototype.}


\begin{figure}[ht!]
\centering
\includegraphics[width=0.8\textwidth, angle=-90]{"images/II/prototype_2.JPG"}
\caption{Initial prototype.}
\end{figure}

\subsection*{testing results}

Due to time constraints, we chose to test the wave machine close to home instead of at the proposed site. Installation went well; it was easy to attach the base securely to the street post and then adjust it to be out of the way of the crosswalk signals. Pulling the handle was satisfying and the motion of the mechanism was smooth and effective. It was fairly visible from across the street when the arm was waving. \href{https://drive.google.com/file/d/1Gzw53p0xpSqZ-RktlmBlWZ7QyR_WPB5E/view?usp=sharing}{Link to video of installed protptype.}

\begin{figure}[ht!]
\centering
\includegraphics[width=0.5\textwidth, angle=-90]{"images/II/install_closeup.JPG"}
\includegraphics[width=0.5\textwidth, angle=-90]{"images/II/install_closeup_2.JPG"}
\caption{Installed mechanism.}
\end{figure}

\begin{figure}[ht!]
\centering
\includegraphics[width=0.8\textwidth, angle=-90]{"images/II/install_far.JPG"}
\caption{Installed mechanism.}
\end{figure}

Unfortunately, while the chosen time and location yielded a high degree of foot traffic, few noticed the wave machine. A few people looked at it, but no one pulled the handle. One person waved back at us while we were using the machine and a few others smiled. Other than that, the initial testing was successful for mechanical refinement, but not so much for user feedback.

\subsection*{next steps}

It was clear after initial testing that the wave machine needed to be more noticeable and the chosen location needed to be more heavily trafficked by individuals out looking for fun. In addition to making the mechanism robust enough to stand up to real use, three modifications were decided upon: improved location, more visible arm, more alluring handle.

We happened to witness a particular crosswalk that was very crowded on both sides late at night. Nearly all of the people on both sides of the street were going to or from bars and were already pretty intoxicated. Many people near the crosswalk were waiting in line to enter various bars. Other than trying to get to the line, none of these people were in a particular rush. 

\begin{figure}[ht!]
\centering
\includegraphics[width=0.3\textwidth]{"images/II/elwire.jpg"}
\caption{Electroluminescent wire.}
\end{figure}

\begin{figure}[ht!]
\centering
\includegraphics[width=0.8\textwidth]{"images/II/essex_magicianbar.JPG"}
\\
\includegraphics[width=0.8\textwidth]{"images/II/essex_opposite.JPG"}
\caption{Nightlife on both sides of proposed crosswalk.}
\end{figure}


Using electroluminescent (EL) wire we planned to increase the visibility of the arm. The EL wire would be placed around the outline of the arm and hand and could be set to be constantly on, or to blink. In addition to grabbing peoples attention, this would create an interesting visual effect as the arm waved through the air.

Lastly, the handle needed to be improved upon. To make it appealing at night we though it would be a good idea to have it light up and blink.



\clearpage
\section*{PART III - Installation}


\end{document}
