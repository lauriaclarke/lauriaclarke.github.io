\documentclass[11pt]{report}
\usepackage{geometry}
\geometry{letterpaper, top=1in, bottom=1in, left=1in, right=1in}                                 
\usepackage[parfill]{parskip}   
\usepackage{graphicx}
\usepackage{amssymb}
\usepackage{epstopdf}
\usepackage{array}
\usepackage{tabularx}
\usepackage{hyperref}
\hypersetup{
    colorlinks=true,
    linkcolor=blue,
    filecolor=magenta,      
    urlcolor=cyan
    }


\setlength{\parindent}{0em}
\pagenumbering{gobble}

\DeclareGraphicsRule{.tif}{png}{.png}{`convert #1 `dirname #1`/`basename #1 .tif`.png}

\newcommand{\newday}[1]{\newpage{\LARGE \textbf{#1}}\\}
\newcommand{\imagefolder}{/Users/lauriaclarke/Documents/mfadt/ms1/lauriaclarke.github.io/instructionsetsforstrangers}


\title{Major Studio I \\ Final Project}
\date{\today}
\author{Lauria Clarke}
                                   
\begin{document}
\maketitle
%\tableofcontents
%----------------------------------------------
\section*{the idea}

The Nature of Breath is a machine that breathes where the speed, depth, and "wetness" of the breathing varies in response to COVID 19 data.

Since COVID started we have all experienced a global fascination with breath. 


It should be naturalistic in the auditory component and artificial in the visual component.

\section*{mechanical breakdown}

Thinking about the way this device will function mechanically is the hardest part since there are few --- if any -- existing examples. To help think about the systems in a more compartmentalized way, I have broken it wodn into three sub systems: actuation, storage, and modulation. Actuation is the mechanism taht forces air into or outof the system, storage is where air is stored between breaths, and modulation is the subsystem which changes the flow of air and produces the appropriate sound. 

In the human body, these three pieces combine to produce breath. 

\subsection*{actuation}

In the body, the thing that actuates airflow through the respiratory system is the diaphragm. It contracts causing air ot be firces in or out of the lungs. 

Actuation could be achieved in a two ways in this project. First, a changing pressure differential forces air in and out of storage medium. This could be accomplished by changing the pressure in a sealed volume which contains the storage comoponent by either using a diaphragm or directly changing the pressure using a pump. Second, the actuation could be direct -- a bi-directional pump forces air through the airway without any kind of storage component. This solution reduces complexity, but may increase cost.

Either way, the acuation technique needs to be bidirectional -- inhale and exhale.

\subsection*{storage}

Storage isn't necessary if actuation is direct on the airway. For visual purposes however, it would be quite striking to see a shape inflating and deflating with each breath. 

\subsection*{modulation}

Modulation is the most complicated and exciting part of this project. It can be furthe briken down into the trachea, larynx, and mouth/sinuses. Each of these three parts of the respiratory system have an impact on how our breating sounds. 

In this system, the trachea will be the first point of focus. The trachea consists of C shaped pieces of cartiladge stack on top of each other with a smooth muscle closing the C. When the smooth muscle contracts the trachea grows more narrow allowing us to do things like cough. 

% include image of trachea here

This mechanism could pretty simple to replicate. 


\section*{initial material gatering and prototypes}

I was quite consumed by this idea after having it, but my ideas for implementation were too flimsy and became quite consuming. I called a number of places that sell balloons and 5 gallon water jug hoping ot find tha materias needed to replicate a gigantic version of that gradeschool experiment. Didn't have much luck. I still couldn't stop thinking about all this, however, so I went to Canal Rubber Co. to look at sheets of latex. This was a GREAT idea. It was soo cool! I gathered more materials than I actually needed and probably spent more than I should've -- \$30.

Then I got home and started playing with the materials and rememebred I'm a bit allergic to latex powder. Ooops!

I build a simple diaphragm test. 


\clearpage
\section*{midterm response}

Hi Harpreet, I neglected to read the prompt very thoroughly for this assignment, so what follows is a very unpolished assesment of my current thinking here. More of a journal entry than anything else. My apologies.

After being here (in MFADT) for a couple months I belive I am persuing the following learning goal: to know when to engage in blindly intuitive making and when to engage in research based, highly considered making. So far my life has revolved around intuitive making, an inexplicable itch. Making first and assigning meaning second. I am here, however, to be challenged and that challenge is to become intentional.

On Wednesday, Franco and I went out and had a beer after class. It felt a bit irresponsible from a time management perspective, but it was really helpful to talk about the midterm process totally outside of school while it was so fresh. The bulk of our conversation focused on language. One aspect of Franco's presentation which I really appreciated was the framework he developed to decribe what he was discussing. For me, language is a big focus right now as I think about what it means to be intentional in my work. In order to be intentional we need to posses the language to describe our intent. Finding the appropriate language to use to describe what I'm seeing and trying to achieve feels like the next, most important step in my process.  

The easiest place to look for language is in what others have already written. I spent some time looking for this during the year before I came to school, but without knowing the languae to look for and use in my search it was difficult to find a path of work to follow. This was frustrating, because I'm usually pretty good at self directed research and reading. Anyway, I'm definitely in the right place now with the right people to guide me in asking questions and building a foundation of thought for finding answers. Language allows you not only to describe, but question a field. 

On the topic of building foundation, I spoke with Tega Brain two weeks ago. I was extremely nervous and practiced a lot. It was a great conversation and she was very generous with her time and knowledge. (As and aside, I am so incredibly excited and grateful be in this place where everyone is willing ot speak with me and discuss the things I'm interested in! I have felt out of place for so much of my academic and professional life it is wonderful to leave taht feeling behind.) I first asked her what frameworks she had encoutnered over her career which have helped lend more specific language and structure to the way the natural and manmade worlds are described and designated. She provided me with many useful references that address that broad ideas I'm trying to learn about. 

The conversation naturally flowed into the next question I wanted to ask which was about the similarities between natural optimization processes and manmade or technological optimization procesed. We agreed that the scale of time and the objective of optimization tend to be the distinguishing factors between the two. In the same vein, she described an overarching urge in her work to replicate manmade systems in natural ways, often relying on this idea of optimization. It was helpful to hear that specific sense of purpose stated in such plain language.

Finally I asked her about the feeling of uncertainty which I am trying to investigate which I fell exists in a lot of her work. She asked if I was trying to describe the uncanny valley. I don't think I am, at least not ``the'' uncanny valley. I believe that the unvanny vally is an appropriate technique to describe what I'm talking about, it is accurate in reflecting the feeling of unease, but is specifically describes a human centric condition. What about more broad, natural existence? That is what I'm after. Maybe someone with a better understanding of this question will tell me that the question itself is unnecessary and has already been answered, but that's for me to find out on my own. 

Thinking a bit about the review itself, I found the process of preparing the presentation much more helpful than delivering the presentaion. Feedback was certainly useful, but I think I could've structured my presentation to make it more helpful. By framing a few specific questions for the audience to ponder I could've directed the feedback. Definitely need to rememebr that next time. Coming up with questions for others to asses for you impllies that know what help you need with so that you can ask for it. A little circular by definition. 

I really liked mark's suggestion of framing the project in diferent ways to understand it better. I'm not sure I entirely know how to go about doing that, but I'm definitly going to try. Richard's coments about following my intuition toward this vein of inquire was helpful in that it made me think more specifically about the sometimes conflicting roles of intuition and intention in this process. The thoughful feedback from class was overwhelming in how geenrous it was (again, feeling lucky to be here)! In many of my classmates' comments I head the same question, which Tanve said really well: what do you want people to take away from your work? Again, a suggestion to focus on my intent.

Overall, the focus on building research and background before engaging in making has made me a lttle bit wary of my own ideas. I think this new degree of consideration is good, but I also recognize that it's important to let an urge to create drive your forward momentum. As I grow as thinkiner and come to understand the space I'm trying to exist in, my process will naturally become more considered. But I've got to start somewhere, and that's right now with the excitement and ideas I have in this moment. 


There is a lot of reading ahead of me. Below is an accumulated list from the last month or so.

\section*{accumulated reading list and internet suggestions}

Biomimicry - Janine Benyus (my dad...who has a weird knack for giving me important books at critical moments, like the Toaster Project right when it came out during my senior year of highschool)

seeing is forgetting the name of the thing one sees - Lawrence Welschler (Ed Andrews)

Anthropocene Thesis (not actually the name) - Donna Haraway (Tega Brain)

Against the Anthropocene - T. J. Demos (Tega Brain)

Exhallation - Ted Chang (Tega Brain)

The Sciences of the Artificial - Herbert Simon (Mark ?)

Life's Edge - Carl Zimmer (me)

Making and Being - 


\vspace{1cm}
Internet followups from Tega Brain:

\url{http://www.greyroom.org/issues/68/72/the-smartness-mandate-notes-toward-a-critique/}

\url{https://alanwarburton.co.uk/fairytales} 

\url{https://jods.mitpress.mit.edu/pub/lewis-arista-pechawis-kite/release/1}

\url{http://www.madlab.cc/}

think this one is the donna haraway\\
\url{https://www.e-flux.com/journal/75/67125/tentacular-thinking-anthropocene-capitalocene-chthulucene/}



\vspace{1cm}
Internet followups from Harpreet:

look at ISEA 2021

\url{https://www.ianingram.org/machines/2009_squirrel.html}

\url{https://www.google.com/search?q=bodies+in+motion&rlz=1C5CHFA_enUS719US719&sxsrf=AOaemvI9jBtAv206EcJ4FUexQurm74VNBw:1633968526778&source=lnms&tbm=isch&sa=X&ved=2ahUKEwiJ7vKM38LzAhXrQ98KHbZjBGoQ_AUoAnoECAIQBA&biw=1440&bih=766&dpr=2}

\url{https://vimeo.com/220895476}

\url{https://news.mit.edu/2018/creating-3-d-printed-motion-sculptures-from-2-d-videos-mit-csail-0919}

\url{https://vvvv.org/blog/bodies-in-motion-humanscale-milan-design-week-2019#:~:text=Bodies%20in%20Motion%20is%20an,scientific%20approach%20to%20furniture%20design}

\url{https://www.youtube.com/watch?v=_v3J6D6R_nI&ab_channel=HumanscaleHQ}

philip beesley

\url{https://www.christianhubert.com/writing/assemblage} 
      
\url{https://link.springer.com/chapter/10.1057/9781137025012_4}



\vspace{1cm}
Internet followups from Richard The:

Design Triennale, Broken Nature

RadioLab -- Breathing podcast

\url{https://www.lozano-hemmer.com/last\_breath.php}


\vspace{1cm}
from Guin: \url{https://www.asianstudies.org/publications/eaa/archives/history-and-sustainability-of-bunraku-the-japanese-puppet-theater/}

from Tanve: The Year the Earth Changed - David Attenborough
%random 
%Building a foundation of language upon which to base your work is critical.
%
%Before coming ot Parsons my practice was veyr much in the vein of making first and assigning meaning second. At present I feel as though I'm being pushed to be more intentional in all areas of my making. 
%
%
%BUIlding a framework or background of language also allows you to call upon references or the prior work in your field more easily. 
%
%I am missing the critical theory that supports the more intuitive exploration I've done so far. I have been looking for it, but without knowing the languae to use in my search it has been difficult to find on my own. A good reminder too that language allows you not only to describe, but question a field. 
%
%I value the feedbak
%
%talk about approaching the work from multiple angles
%I have a hard time letting go of my desire to solve physical challenges through making. This particular desire has very deeply rooted in me for as long as I can remember. Coming up with a physical solution to a complex and often unecessary problem is what drive a lot of my work. 
%
%Again, I think that I need to asses this urge through the lens of intentionality. It can be useful at time, but learning somplementary ways of thinking about the intent or goal of my work is a skil which I really hope to develop in my time here at parsons.
%
%I spoke with tega brain last week and that was a fantastic conversation. She provided me with many, many references that address the questions in her own work and mine. The conversation we had about the specific questions I asked her felt really fruitful as well -- I should've typed this up sooner. We spoke about the paralles between optimization in nature and optimization in engineering and more.
%
%
%chat with Tega Brain
%
%chat with Rob
%
%
%When Mark asked my why ths question of defining what is natural vs. artificial matters I felt like I stumbled a bit. The answers that usually run through my head felt wrong for the moment, but I need to get better at saying what I mean. I have this idealized scenario where a kid grows up in a house where the deck outside is made of fake wood and all the floors in the house are made of fake wood and every single piece of wood around is actually plastic - exactly like Through the Arc of the Rainforest(!). This kid never know what real wood is. Wouldn't that be problematic? To be so out of touch. Maybe I need to re-read through the arc of the rainforest for better insight on this one. Although everyone dies at the end of that book so maybe it's not really so helpful.
%
%
%
%
%suggestion from tanve: what do you want people to take away from your work?

\clearpage
\section*{chat with harpreet on 10/25}

7 in 7 is an attempt to try multiple things in many ways

use different perspective to address the same question

could use 

is the conecptual prototype having the intended effect on users 

are the users having the intended experience  

\clearpage
\section*{{\LARGE7 in 7} (in reverse order, accompanying photos forthcoming)}
also, spellcheck in my text editor isn't working for this document


\newpage
\subsection*{five (11/03/2021)}
\textbf{name:} trachea support trial no. 1

\textbf{soundtrack:} Empire of the Sun




\newpage
\subsection*{four (11/03/2021)}
\textbf{name:} latex / trachea trial no. 2

\textbf{soundtrack:} Gillian Welch discography all day (every couple months I spend one day listening to all her music) 

The intent of this prototype was to see how close to a trachea structure I could get. I repeated the same experiemnt from Tuesday night more carefully and with more components in a nattempt to construct ad better trachea protptype. The goal was to create a closed tube with a mechanism in place to create a change in shape.

First I carefully cur my pieces of latex to size -- two 3" x 1.5" pices. Then I cut four pieces of vinyl tubing to 1.5" lengths to embed in the latex. I cleaned and prepared the latex with glue, then sandwiched the vinyl tubing between the latex pieces. I used a pretty thin and even layer of glue and was very careful to prevent air bubbles. 

Next I took another 1.5" wide piece of latex and applied glue to the edge. I applied glue to one edge of my tubing assembly and after waiting the appropriate aount of time, carefully stuck the two edges together with about 1/4" of overlap. 

The next step -- closing the tube -- was tricky. To prevent the vinyl tubing from pulling apart the new seams I took pieces of wire and threaded them through the tubing. Then I bent them into a U shape keeping the twp sides I waas gluing close together with little tension on them. I then applied glue to the other side of the sandwich andt he other isde of the flap and sealed the other side. 

After letting this sit for some time, I began to remove the wire pieces and replace them with bits of monofilament like in the previous prototype. I was pleasantly surprised by how well the glue sustained the tension from the vinyl tubing. Nothing began to tear or separate. 


Things I learned:
\begin{itmemize}
\item
  this was pretty a easy easy with a 3" section, but I think it will be more difficult with a longer tube
\item
  the latex to latex bond is pretty strong and didn't come apart from the pressure of the viny tubing
\item
  I need to think of a clever way to mount this so that it can be mechanised
\item
  
\end{itemize}


\newpage
\subsection*{nothing (11/02/2021)}

Had to do work work for my job instead.


\newpage
\subsection*{three (11/01/2021)}
\textbf{name:} latex trial no. 1

\textbf{soundtrack:} my own breath in a respirator

This prototype is also going to span multiple days and build on itself. The intent of this first try was to get a feel for working with latex. 

First I assembled my materials. A few 2" x 2" squares of 14 mil latex, some rubbing alchohol, and a bottle of rubber cement. Then I prepared my work area, which was a little tricky since the window in my studio wouldn't open. I ended up working on top of the stove with the vent hood on while weraing a respirator.

After reading a little bit on the internet I went for it. The order of operations is to put glue on both pieces of latex, let it dry for about 5 minutes and then stick the two together. When you put the glue on initially the latex shrivels up. As time passes it relaxes and returns to its original shape. Even though I expected this I was pretty surprised to see just how much it shrivled. I wasn't really sire about the rubbing alchohol so I left it out at first. 

The first test I did turned out well, all things considered. The two pieces of latex were firmly stuck together. While the latex was relaxing I started watching a youtube video for more instruction. Even though I hate instructional youtube videos, this one was quite helpful. I learned that I was using too much glue and that it's a good idea to wipe the latex with rubbing alchohol before applying glue, as I suspected.

For the second test I used a q-tip and first wiped the gluing area with isopropyl. Then, using another q-tip I applied a much thinner layer of glue than before. The latex still curled up, but it was less messy and uncurled faster. I was also more mindful to avoid air bubbles as I stuck the two pieces together. This test went pretty well, but still had a lot of air bubbles and wasn't very flat.

For the last test I decided to embed two small pieces of vinyl tubing between the sheets of latex to play with changing the shape of the sheet. I applied the glue as usual and then once it had dried set the two pieces of tube on the glue side oasdf one piece. I then carefully applied the other piece of glue latex. This time I was veyr mindful not to get any airbubbles and the final result was quite smooth. Its appearnce is certainly reminiscent of ridged tissue with cartalidge.

Then I threaded some monofilament through the tube and some glass beads I found in the dryer with my laundry -- the fact that I can always seem to find the materials I need, but not my fancy new bike sock is irksome but feels fair -- and played around. Without a structure to hold the tube it was tricky to contract it. 



Things I learned and new questions to address:
\begin{itemize}
\item
the british guy on youtube said you should only work with a small vessel of gue and isopropyl at a time to minimize evaporation...need to find some tiny jars
\item
need to determine how strong the bond is -- how could I create an anchor point needed to mechanize the diaphragm?
\item
my experiement with the vinyl tubing is cool, but it would be better if the vinyl formed a closed tube and the whole thing was more mechnically robust
\item
what sounds or feeling does water have on the latex?
\item
I continue to have a slight allergy to latex powder. Hopefully it doesn't get worse. I will find nitrile gloves and continue not to touch my face.
\item
...more questions to come
\end{itemize}


\newpage
\subsection*{two (10/31/2021)}

\textbf{name:} disembodied breath no. 1

\textbf{soundtrack:} some of Mirrors by Angel Olsen and Late Night Feelings by Mark Ronson et. al ...piecemeal listening today

This prototype is defintely going to span two days. Today being Halloween seemed like an appropriate day to start, however.

The goal of this prototype is to understand the response to the sound of disembodied breath. To achieve this will use my trusty buckets to "produce" breath sounds. I would say that this prototype is Wizard of Oz adjacent.

Fist I took a small bluetooth speaker and the biggest USB battery I have. I made sure that they fit inside one 5 gallong bucket when another was placed on inside. 

Then I put the bucket from yesterday with the hole and latex tube attached inside the bucket with the speaker. 

I connected to the speaker with my cellphone and played a youtube video of breath noises at a loud volume. It was a very striking effect.

I took my buckets -- the new, hottest trend in streetwear accessories -- out on the street and found a cafe on a pretty busy corner. I put them on the sidewalk and played breath noises.

A number of things went wrong; I needed to use my cellphone for something other than youtube, the bluetooth kept disconnecting, and due to street noise and music from the cafe no one really heard the bucket. I also placed it near a trash can and was worried someone would throw something yucky inside.

Then I went home and created an mp3 from the youtube video. I put it on an old cellphone on repeat and used an aux cable to connect the speaker to the cellphone. 

Sometime later I took this, much more solid, combination out to my favorite corner bar with my sister who was visiting. And placed it near a crosswalk (the same one I used for instruction sets for strangers, in fact).

Not too many people noticed the bucket. I think the road noise was a lot to contend with. The people that did notice were pretty funny. Three interactions stand out:
\begin{itemize}
\item
A young woman kept looking at the bucket while waiting for the crosswalk with a group of friends. She kept looking furtively over her shoulder at the bucket and was clearly perturbed by the noise. It was also clear from her interaction walking away that she didn't immediately mention it to her companions. 
\item
A man in his late 50s perhaps was eating a slice of pizza while crossing the street. When he got to the bucket he stopped and was very curious. He looked inside and listened closely. He clearly wasn't in a rush as he finished his pizza. Throughout the interaction his facial expression was a mix of confusion, disgust and laughter. He walked away shaking his head. 
\item
A young man was smoking a cigarette near the bucket. After some time he started looking at it more closely and clearly noticed the sound. He seemed confused and a bit disconcerted. My sister and I were clearly looking at him so he came over to us and mentioned that he though someone had ``planted something" on the corner. He said he'd thought it was music and had been waiting for a pop song to start playing, but nothing happened. 
\item
Finally, a woman collecting recycling walked by. She stopped at the trash can adjacent to the bucket. And then I watched in seeminly slow motion as her attention turned to my prized buckets. She seemed a bit confused, but bent over and took the inner bucket out. At that point I left my seat and went over and asked her to stop. I explained that the buckets were mine and she apologized and went on her way. Then it was time ot go home.
\end{itemize}

The big takeaways which I want to adjust on the next iteration aere as follows:
\begin{itemize}
\item
needs to be louder
\item
needs a more quiet location
\item
need a better way to control which breath sounds are playing
\end{itemize}

To account for these, I think I'm going to try a park. I can't find a louder speaker, but maybe I can make the noise louder digitally. I am also going to create simple interface which will let me adjust which breath sounds are playing instead of having to listen to the youtube video mp3 on repeat.
\newpage
\subsection*{one (10/30/2021)}

\textbf{name:} diaphgarm test no. 1

\textbf{soundtrack:} Treehouse by Sofi Tukker

The objective of this prototype was to create a mechanism for forcing air through a tube so that I can play with different sounds. To do this I intended to create a diaphragm to move the air. My intention was to attach a pice of latex over one end of the bucket and a small tub at the other.

I started with one of the 5 gallon buckets I found and bought a bucket lid (should've tried harder to find the lids when I found the buckets).

I cut the center out of the lid and then did a test to see how well the latex stayed in place under the lid. Luckily it worked well and didn't damage the latex.

Next, I drilled a hole in the bottom of the bucket. I forgot about my hole saw kit and just used a knife to slowly shave off the edges until it was the right size for the tube.

Then I trimmed a pice of Latex to size (this required a fresh box cutter since all my other knives were too dull for a clean cut).

Finally, I attached the latex to the open end of the bucket under the "lid" and poked the tub through the hole in the bottom.

The effect was pretty immediate. By depressing the diaphragm I easliy forced air out through the tube -- an exhale -- and air came back into the tube when I let the diaphragm rebound -- inhale. 

There were three pnysical properties I wanted to play with to see how they impacted the sound coming from the prototype:
\begin{enumerate}
\item
how much of the tube was outside of the bucket vs. inside
\begin{itemize}
\item
This didn't seem to have a big impact. It was really ahrd to tell the difference between the tube being all the way out and all the way in. I think I need to cut the tube ot be different lengths.
\end{itemize}
\item
how sound changed when the tube was pinched in different places
\begin{itemize}
  \item
    when the tube was pinched near the bottom the sound was lower in pitch
  \item
    when the tube was pinched near the top the sounds was higher in pitch
  \item
    nearer the top sounded more like human breathing
\end{itemize}
\item
how sounds changed if the bucket was crushed sideways to be less round
\begin{itemize}
  \item
    this one was also hard to tell, it seemed that there was slighly less resonance when the bucket was crushed
  \item
    the intent was to make the sound less hollow and I think this was slightly true
\end{itemize}

\end{enumerate}

\vspace{1cm}
I think this prototype was pretty successful. It left me with a good number of questions which are as follows.
\begin{itemize}
\item
  how does the length of the output tube effect the sound? need to experiement, but an not ready to cut the tube.
\item
  how does the shape of the cavity effect the sound?
\item
  is this cavity ridiculously large in comparison to normal lungs?
\item
  how do I make the sound less hollow?
\item
  what volume of air is displaced? (if I had a flow meter I could just integrate, but I'll have to be more clever)
\item
  how would a lung structure change the sound?
\item
  how would the sound change if there was some kind of foam or cloth inside the cavity? would that sound less hollow?
\item
  how the heck am I going to automate the diaphragm?
\end{itemize}
\newpage
\section*{7 in 7 planning}

The first step of any prototyping exercise is to collect materials. I did some of this a couple weeks ago when I went to canal rubber (so cool, so much sneezy latex powder). This was good fodder for the diaphragm and trachea prototypes, but I still need a chest cavity like thing that would be more robust than the small plastic container I tried already. A large plastic carboy for brewing would be ideal, but that violates my prototyping rule of only buying things you truly need for a material or functional evaluation (like latex) and can't find anywhere else. While out on a late walk this Monday evening I found a lot of recycling outside a restaurant near my apartment. Isolated inside a single clear trashbag were three 5 gallon buckets formerly containing pickles. These streets deliver! They didn't even smell much like pickles once I got them home and opened the bag. After a few days outside and a big rainstorm they were good to go. Seems like I already have the necessary materials for my other physical protptypes. 

\subsection*{proposed prototypes}

There are two sensory aspects of this work, the auditory and the physical or visual. It was pretty easy to identify prototypes that fell in each of these categories. The auditory prototypes should play with differnet types of breath and evaluate the effect of disembodied breath. The physical prototype should focus solely on the mechanical aspec of reproducing breath according to the basic breakdown of the respiratory system I've outlined. I'm not going to start thinking about the electronics used to actuate the thigns.
%outside of investigating what might be quiet.

It was less easy to think about prototypes that address the less tangible questions I am hoping to probe with this work. To think about this I started by looking back at the questions I asked myself originally. After reviewing that list, the big question I want to start asnwering is how a viewer sees their own presesnce near the work impacting its breath. This is an answer to the questions of who holds power over our own breath and how far does breath extend as a metaphor for out relationship with the natural world. I'm going to start with this one isolated question because it has a lot of facets and ultimately will lead to the bigger question of what should the viewer take away from the work (in other words, what experience do I intend for them to have). 

Based on the above thinking I concoted the following protptypes.


%What if as more people enter a space, or come closer to the contraption, it becomes more difficult for it to breath?

\vspace{1 cm}
\textbf{interactive (how a viewer sees their own presence near the work impacting its breath):}

as someone approaches the bucket the breath becomes worse

as someone leaves the bucket breath becomes worse

breath eminates from a tree? that's a bit random, but perhaps illustrative

\vspace{1 cm}
\textbf{auditory (addressing disembodied breath, how breath denotes sickness, difficulty, and emotion):}

Hide a speaker and battery inside two 5 gallon buckets in a public place with moderate traffic and see how people react. Do each of these different types of breath illicit different reactions?

play normal breath sounds 

play laboured and sick (covid equivalent) breath sounds

play emotional breath sounds (sad, happy, aroused, etc.)


\vspace{1 cm}
\textbf{visual / physical (addressing how to actually make this thing and how it should look):}

make a big diaphragm and play with \textit{quiet} ways to make it move

make a trachea protptype with multiple rings that contract

play with the idea of a larynx

model a brachea in openSCAD and print one small piece to think about how moitsure could interact with this piece


\end{document}
